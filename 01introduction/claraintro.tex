% Options for packages loaded elsewhere
\PassOptionsToPackage{unicode}{hyperref}
\PassOptionsToPackage{hyphens}{url}
\PassOptionsToPackage{dvipsnames,svgnames,x11names}{xcolor}
\documentclass[
]{article}
\usepackage{xcolor}
\usepackage{amsmath,amssymb}
\setcounter{secnumdepth}{-\maxdimen} % remove section numbering
\usepackage{iftex}
\ifPDFTeX
  \usepackage[T1]{fontenc}
  \usepackage[utf8]{inputenc}
  \usepackage{textcomp} % provide euro and other symbols
\else % if luatex or xetex
  \usepackage{unicode-math} % this also loads fontspec
  \defaultfontfeatures{Scale=MatchLowercase}
  \defaultfontfeatures[\rmfamily]{Ligatures=TeX,Scale=1}
\fi
\usepackage{lmodern}
\ifPDFTeX\else
  % xetex/luatex font selection
\fi
% Use upquote if available, for straight quotes in verbatim environments
\IfFileExists{upquote.sty}{\usepackage{upquote}}{}
\IfFileExists{microtype.sty}{% use microtype if available
  \usepackage[]{microtype}
  \UseMicrotypeSet[protrusion]{basicmath} % disable protrusion for tt fonts
}{}
\makeatletter
\@ifundefined{KOMAClassName}{% if non-KOMA class
  \IfFileExists{parskip.sty}{%
    \usepackage{parskip}
  }{% else
    \setlength{\parindent}{0pt}
    \setlength{\parskip}{6pt plus 2pt minus 1pt}}
}{% if KOMA class
  \KOMAoptions{parskip=half}}
\makeatother
\usepackage{color}
\usepackage{fancyvrb}
\newcommand{\VerbBar}{|}
\newcommand{\VERB}{\Verb[commandchars=\\\{\}]}
\DefineVerbatimEnvironment{Highlighting}{Verbatim}{commandchars=\\\{\}}
% Add ',fontsize=\small' for more characters per line
\newenvironment{Shaded}{}{}
\newcommand{\AlertTok}[1]{\textcolor[rgb]{1.00,0.00,0.00}{\textbf{#1}}}
\newcommand{\AnnotationTok}[1]{\textcolor[rgb]{0.38,0.63,0.69}{\textbf{\textit{#1}}}}
\newcommand{\AttributeTok}[1]{\textcolor[rgb]{0.49,0.56,0.16}{#1}}
\newcommand{\BaseNTok}[1]{\textcolor[rgb]{0.25,0.63,0.44}{#1}}
\newcommand{\BuiltInTok}[1]{\textcolor[rgb]{0.00,0.50,0.00}{#1}}
\newcommand{\CharTok}[1]{\textcolor[rgb]{0.25,0.44,0.63}{#1}}
\newcommand{\CommentTok}[1]{\textcolor[rgb]{0.38,0.63,0.69}{\textit{#1}}}
\newcommand{\CommentVarTok}[1]{\textcolor[rgb]{0.38,0.63,0.69}{\textbf{\textit{#1}}}}
\newcommand{\ConstantTok}[1]{\textcolor[rgb]{0.53,0.00,0.00}{#1}}
\newcommand{\ControlFlowTok}[1]{\textcolor[rgb]{0.00,0.44,0.13}{\textbf{#1}}}
\newcommand{\DataTypeTok}[1]{\textcolor[rgb]{0.56,0.13,0.00}{#1}}
\newcommand{\DecValTok}[1]{\textcolor[rgb]{0.25,0.63,0.44}{#1}}
\newcommand{\DocumentationTok}[1]{\textcolor[rgb]{0.73,0.13,0.13}{\textit{#1}}}
\newcommand{\ErrorTok}[1]{\textcolor[rgb]{1.00,0.00,0.00}{\textbf{#1}}}
\newcommand{\ExtensionTok}[1]{#1}
\newcommand{\FloatTok}[1]{\textcolor[rgb]{0.25,0.63,0.44}{#1}}
\newcommand{\FunctionTok}[1]{\textcolor[rgb]{0.02,0.16,0.49}{#1}}
\newcommand{\ImportTok}[1]{\textcolor[rgb]{0.00,0.50,0.00}{\textbf{#1}}}
\newcommand{\InformationTok}[1]{\textcolor[rgb]{0.38,0.63,0.69}{\textbf{\textit{#1}}}}
\newcommand{\KeywordTok}[1]{\textcolor[rgb]{0.00,0.44,0.13}{\textbf{#1}}}
\newcommand{\NormalTok}[1]{#1}
\newcommand{\OperatorTok}[1]{\textcolor[rgb]{0.40,0.40,0.40}{#1}}
\newcommand{\OtherTok}[1]{\textcolor[rgb]{0.00,0.44,0.13}{#1}}
\newcommand{\PreprocessorTok}[1]{\textcolor[rgb]{0.74,0.48,0.00}{#1}}
\newcommand{\RegionMarkerTok}[1]{#1}
\newcommand{\SpecialCharTok}[1]{\textcolor[rgb]{0.25,0.44,0.63}{#1}}
\newcommand{\SpecialStringTok}[1]{\textcolor[rgb]{0.73,0.40,0.53}{#1}}
\newcommand{\StringTok}[1]{\textcolor[rgb]{0.25,0.44,0.63}{#1}}
\newcommand{\VariableTok}[1]{\textcolor[rgb]{0.10,0.09,0.49}{#1}}
\newcommand{\VerbatimStringTok}[1]{\textcolor[rgb]{0.25,0.44,0.63}{#1}}
\newcommand{\WarningTok}[1]{\textcolor[rgb]{0.38,0.63,0.69}{\textbf{\textit{#1}}}}
\usepackage{graphicx}
\makeatletter
\newsavebox\pandoc@box
\newcommand*\pandocbounded[1]{% scales image to fit in text height/width
  \sbox\pandoc@box{#1}%
  \Gscale@div\@tempa{\textheight}{\dimexpr\ht\pandoc@box+\dp\pandoc@box\relax}%
  \Gscale@div\@tempb{\linewidth}{\wd\pandoc@box}%
  \ifdim\@tempb\p@<\@tempa\p@\let\@tempa\@tempb\fi% select the smaller of both
  \ifdim\@tempa\p@<\p@\scalebox{\@tempa}{\usebox\pandoc@box}%
  \else\usebox{\pandoc@box}%
  \fi%
}
% Set default figure placement to htbp
\def\fps@figure{htbp}
\makeatother
% definitions for citeproc citations
\NewDocumentCommand\citeproctext{}{}
\NewDocumentCommand\citeproc{mm}{%
  \begingroup\def\citeproctext{#2}\cite{#1}\endgroup}
\makeatletter
 % allow citations to break across lines
 \let\@cite@ofmt\@firstofone
 % avoid brackets around text for \cite:
 \def\@biblabel#1{}
 \def\@cite#1#2{{#1\if@tempswa , #2\fi}}
\makeatother
\newlength{\cslhangindent}
\setlength{\cslhangindent}{1.5em}
\newlength{\csllabelwidth}
\setlength{\csllabelwidth}{3em}
\newenvironment{CSLReferences}[2] % #1 hanging-indent, #2 entry-spacing
 {\begin{list}{}{%
  \setlength{\itemindent}{0pt}
  \setlength{\leftmargin}{0pt}
  \setlength{\parsep}{0pt}
  % turn on hanging indent if param 1 is 1
  \ifodd #1
   \setlength{\leftmargin}{\cslhangindent}
   \setlength{\itemindent}{-1\cslhangindent}
  \fi
  % set entry spacing
  \setlength{\itemsep}{#2\baselineskip}}}
 {\end{list}}
\usepackage{calc}
\newcommand{\CSLBlock}[1]{\hfill\break\parbox[t]{\linewidth}{\strut\ignorespaces#1\strut}}
\newcommand{\CSLLeftMargin}[1]{\parbox[t]{\csllabelwidth}{\strut#1\strut}}
\newcommand{\CSLRightInline}[1]{\parbox[t]{\linewidth - \csllabelwidth}{\strut#1\strut}}
\newcommand{\CSLIndent}[1]{\hspace{\cslhangindent}#1}
\setlength{\emergencystretch}{3em} % prevent overfull lines
\providecommand{\tightlist}{%
  \setlength{\itemsep}{0pt}\setlength{\parskip}{0pt}}
\usepackage{setspace}
\usepackage[left=2.2cm, right=2.2cm, top=2cm, bottom=2.3cm]{geometry}
\setlength{\skip\footins}{0.7cm}
\usepackage{nowidow}
\widowpenalty=10000
\clubpenalty=10000
\makeatletter
\renewcommand{\footnoterule}{%
  \kern -3pt
  \hrule width 0.35\linewidth
  \kern 8pt
}
\makeatother
\usepackage{bookmark}
\IfFileExists{xurl.sty}{\usepackage{xurl}}{} % add URL line breaks if available
\urlstyle{same}
\hypersetup{
  pdftitle={Pour un workflow de révision -- introduction explicative},
  pdfauthor={Clara Grometto},
  colorlinks=true,
  linkcolor={blue},
  filecolor={Maroon},
  citecolor={blue},
  urlcolor={blue},
  pdfcreator={LaTeX via pandoc}}

\title{Pour un workflow de révision -- introduction explicative}
\usepackage{etoolbox}
\makeatletter
\providecommand{\subtitle}[1]{% add subtitle to \maketitle
  \apptocmd{\@title}{\par {\large #1 \par}}{}{}
}
\makeatother
\subtitle{FRA 6730}
\author{Clara Grometto}
\date{décembre 2025}

\usepackage{scrextend} 
\KOMAoptions{footnotes=multiple}
\addtokomafont{footnote}{\fontsize{9.5pt}{10.5pt}\selectfont}
\newcommand*\footnotemarkspace{2em} 
\deffootnote{\footnotemarkspace}
  {\parindent}
  {\makebox[\footnotemarkspace][r]{\fontsize{9.5pt}{10.5pt}\selectfont\thefootnotemark.\hspace{0.8em}}} 
\makeatletter
\renewcommand{\@makefnmark}{\hbox{\hspace{0pt}\textsuperscript{\normalfont\@thefnmark}}}
\makeatother

\begin{document}
\maketitle

\doublespacing

Ce travail constitue une expérimentation-réflexion autour de la
modélisation de la révision éditoriale, notamment dans le cadre de
l'édition scientifique. Les sources sont les versions de travail de
trois articles de la revue \emph{Sens Public}, récupérées sur l'éditeur
collaboratif en ligne Stylo, qui permet le versionning, et donc de
remonter à des états antérieurs du texte. Il s'agit non seulement de
modéliser les types de correction, mais aussi le processus dialogique
qui s'engage entre les divers·es acteur·rices. Nous voudrions dépasser
une approche strictement typologique des révisions pour les intégrer à
la dynamique de leur réception et de leur discussion, en nous inspirant
des gestes et des pratiques des éditeur·rices.

Aujourd'hui Stylo montre un défaut\,: l'invisibilisation du travail
éditorial. La fonctionnalité de comparaison des versions était
insatisfaisante car elle manquait de lisibilité, et a été supprimée dans
le but d'être améliorée. De manière générale, le dialogue entre
éditeur·ice et auteur·ice tend à se déplacer vers des outils externes.
Cela se traduit par une fragmentation des échanges-- multiplication de
fichiers, de commentaires, de courriels -- qui affaiblit la cohérence du
processus éditorial et souligne l'impensé du workflow. Dans Stylo, cette
limite est partiellement contournée par l'intégration d'Hypothesis, qui
cependant ne permet pas une modélisation des interventions sur le texte.
C'est précisément là que réside la force du \emph{Track Changes} de
Word\,: il rend le travail des correcteur·rices immédiatement visible,
traçable et négociable, et ce dans un seul fichier.

L'enjeu est donc de concevoir un dispositif qui permette à la fois de
mieux structurer l'échange entre auteur·ice et correcteur·ice, de rendre
visible le travail éditorial, et ce sans renoncer à une modélisation des
différents types d'interventions. Ce travail constitue d'ailleurs une
réflexion sur la pertinence des différents types d'intervention à
visibiliser en priorité.

Les premiers entretiens informels menés avec les éditeur·rices de
\emph{Sens Public} et de \emph{Humanités Numériques} montrent que la
classification des révisions ne peut être pensée indépendamment des
métiers et des expertises mobilisés. La première distinction qui
paraîtrait aidante relève du degré d'\,``importance'' des corrections.
Ce terme, employé par Florence Daniel (éditrice de \emph{Humanités
Numériques}), revêt selon nous une connotation hiérarchisante. Nous
n'emploierons pas ce terme afin d'éviter de distinguer les types de
correction en termes de niveau mais plutôt en termes de domaine
d'expertise mobilisé.

Les corrections orthotypographiques et grammaticales, les vérifications
de fond et de forme sur la bibliographie, ou l'assurance d'un rendu
final conforme relèvent du domaine des correcteur·rices et
éditeur·rices. Elles ne sont pas moins importantes et participent
également à la production du sens, d'ailleurs les auteur·rices doivent
en prendre connaissance, mais ils doivent les recevoir comme des
corrections de la part d'experts de la langue. Les reformulations plus
substantielles -- déplacements de segments, suppressions de redondances,
reformulations visant la clarté -- engagent davantage la structure du
propos ou bien le style auctorial, et prennent alors la valeur de
suggestions ou de questions adressées à l'auteur·ice.

Prolongeant les pratiques de l'édition papier (dans lesquelles il était
d'usage d'indiquer par des signes normalisés la nature précise des
corrections) dans l'environnement numérique, nous aimerions aussi
superposer une catégorisation plus fine. Pour élaborer cette typologie,
nous nous sommes appuyée sur les distinctions formulées dans le Chicago
Manual of Style (18ᵉ édition),\footnote{\citeproc{ref-ChicagoManualStyle}{{``The
  {Chicago Manual} of {Style}, 18th {Edition}.''} URL~:
  \url{https://www.chicagomanualofstyle.org/book/ed18/frontmatter/toc.html}}}
qui propose un état des lieux très complet des étapes du processus de
publication et d'édition, et des différentes règles orthotypo à
respecter. Prenant chacune des catégories du sommaire, nous avons tenté
de les factoriser pour avoir un nombre plus limité de types.

Nous avons travaillé directement sur la syntaxe Markdown considérée
comme du texte brut, car les éditeur·rices et auteur·rices de Sens
Public produisent et corrigent directement les articles dans ce format.
Une part significative du travail des éditeur·rices consiste précisément
à intervenir sur le balisage Markdown, et non uniquement sur le contenu
linguistique.

Voici donc la double catégorisation appliquée\,:

\begin{enumerate}
\def\labelenumi{\arabic{enumi}.}
\tightlist
\item
  Corrections mécaniques\,: mécanique de la langue, mécanique du code,
  mécanique de la pipeline, fluidité de la lecture
\end{enumerate}

\begin{itemize}
\tightlist
\item
  orthographe -- rectifications d'erreurs d'orthographe, harmonisation
  des graphies.
\item
  Casse, capitalisation, abréviations, ponctuation, faces -- questions
  d'uniformisation des majuscules/minuscules, des abréviations, des
  signes de ponctuation, espaces insécables. Ces changements peuvent
  éventuellement se faire en masse.
\item
  Numérotation, listes, dates, chiffres -- correction ou cohérence des
  éléments numériques.
\item
  Grammaire et Syntaxe -- accord, conjugaison, structure de phrases,
  impropriétés syntaxiques.
\item
  Code markdown -- harmonisation et correction pour rentrer sans
  frottement dans la chaîne de conversion. Dans cette catégorie peut
  rentrer la correction du BibTeX et de l'insertion de la citation dans
  le corps de texte. Ici, le rendu présentationnel côtoie le sémantisme
  des données.
\end{itemize}

\begin{enumerate}
\def\labelenumi{\arabic{enumi}.}
\setcounter{enumi}{1}
\tightlist
\item
  Modifications énonciatives ou discursives (assumant la position de
  l'auteur)\,: affectent l'organisation du discours, la rhétorique, le
  style, la voix de l'auteu·rice et impliquent un dialogue plus proche
  avec celleux-ci.
\end{enumerate}

\begin{itemize}
\tightlist
\item
  ponctuation -- virgules, points, tirets, guillemets, usage des
  italiques avec une incidence sur l'argumentaire ou le style.
\item
  Substitutions lexicales / de genre / de nombre avec une incidence sur
  l'argumentaire ou le style.
\item
  Ajouts ou suppressions de mots/phrases.
\item
  Déplacements de mots/phrases.
\item
  Réécriture ponctuelle de segments, ou plutôt proposition de
  reformulation, pour des raisons autres que grammaticale ou syntaxique.
\end{itemize}

\begin{enumerate}
\def\labelenumi{\arabic{enumi}.}
\setcounter{enumi}{2}
\tightlist
\item
  Autour du texte -- interventions sur les références et paratexte
  éditorial
\end{enumerate}

\begin{itemize}
\tightlist
\item
  Ajouts de notes de bas de page éditoriales.
\item
  Correction de citations.
\item
  Vérification de fond sur la bibliographie, les figures et les tableaux
  insérés.
\end{itemize}

Nous proposons la modélisation d'un workflow de révision basé sur la TEI
pour l'alignement de deux versions d'un article, et plus largement sur
les technologies XML afin de rendre visibles, pour l'auteur, les
corrections apportées à la version révisée. Le workflow repose sur une
distinction claire des rôles\,: un premier acteur effectue les
révisions, tandis qu'un second acteur en prend connaissance et en valide
-- ou refuse -- certaines.

Le recours à XML peut sembler lourd ou peu adapté, et il est vrai que la
chaîne mise en œuvre pour arriver au prototype final est assez complexe
et pour le moins artisanale. Néanmoins, la TEI a l'avantage d'être
conceptuellement très expressive, offrant un modèle déclaratif qui va
permettre d'encoder finement les phénomènes textuels et les opérations
éditoriales. Ce choix se justifie aussi par des intérêts très personnels
et la volonté de découvrir et de nous essayer à XForm.

Le choix des balises \texttt{\textless{}choice\textgreater{}},
\texttt{\textless{}sic\textgreater{}} et
\texttt{\textless{}corr\textgreater{}} s'appuie sur les guidelines
(3.5).

\begin{quote}
As in editing a printed text, so in encoding a text in electronic form,
it may be necessary to accommodate editorial comment on the text and to
render account of any changes made to the text in preparing it. The tags
described in this section may be used to record such editorial
interventions, whether made by the encoder, by the editor of a printed
edition used as a copy text, by earlier editors, or by the copyists of
manuscripts.\footnote{\citeproc{ref-TEIGuidelinesSimple}{{``{The TEI
  Guidelines - Simple Editorial Changes}.''} URL~:
  \url{https://tei-c.org/release/doc/tei-p5-doc/en/html/CO.html}}}
\end{quote}

Ce set de balises semblait bien plus pertinent que le détournement des
éléments dédiés à un apparat critique (comme il était question dans la
note d'intention). Pour simplifier la pipeline, nous nous sommes limitée
à ces trois balises. Les guidelines distinguent les corrections
d'erreurs évidentes de la normalisation ortho-typo (moins universelle),
des additions, suppressions, et des substitutions. Nous nous appuierons
plutôt sur l'attribut \texttt{@corresp} de l'élément
\texttt{\textless{}choice\textgreater{}} pour qualifier plus précisément
l'intervention encodée.

\subsection{Description du workflow}\label{description-du-workflow}

La première étape, à partir de l'export du markdown depuis Stylo, est le
nettoyage des sources brutes. Essentiellement, cela consiste à supprimer
les retours chariot qui ne correspondent pas à la création d'un nouveau
paragraphe en Markdown, afin de supprimer les différences non
significatives entre les deux versions du texte et de faciliter le
passage de l'algorithme de \emph{diff}. Les doubles retours, qui
signalent un paragraphe en Markdown, sont quant à eux remplacés par les
balises
\texttt{\textless{}milestone\ unit="tei:p"/\textgreater{}\textless{}lb/\textgreater{}}\footnote{Méthode
  suggérée par \textsc{Beshero-Bondar}, Elisa,
  \citeproc{ref-beshero-bondarDocumentModelingTEI2019}{{``Document
  {Modeling} with the {TEI Critical Apparatus}: {A Panel
  Presentation}.''} September 2019. }}, encore une fois pour améliorer
l'alignement les deux versions. En effet, le changement de paragraphe
est une variante significative qu'il faut prendre en compte. Ces deux
premières étapes sont effectuées grâce à de simples chercher-remplacer.

Ensuite, un \emph{diff} est généré grâce à l'algorithme \texttt{wdiff},
qui considère le mot comme la plus petite unité de comparaison. Le
résultat est stocké dans un fichier \texttt{comparaison.txt}, lequel est
traité par un script Python (\texttt{comparaison.py}). Il identifie, à
l'aide d'expressions régulières, les segments supprimés et ajoutés,
balisés respectivement \texttt{{[}-\ …\ -{]}} et \texttt{\{+\ …\ +\}}.
Il réinjecte ces contenus dans des éléments
\texttt{\textless{}choice\textgreater{}},
\texttt{\textless{}sic\textgreater{}} et
\texttt{\textless{}corr\textgreater{}}, puis nettoie les espaces
superflus (espaces résiduels partagés par les deux versions, sans
incidence sur la traçabilité des différences). Le résultat est intégré
dans un squelette TEI comprenant la base d'un header.

Le fichier XML obtenu est ensuite repris manuellement. Cette étape
permet de vérifier tout le texte et de compenser les limites du script,
qui se révèle efficace pour les cas généraux (les substitutions, les
plus fréquentes) mais échoue dans certains cas spécifiques (par exemple
l'ouverture d'un élément \texttt{\textless{}choice\textgreater{}} en
rencontrant une syntaxe Markdown de type \texttt{{[}-@refId{]}}). Aussi,
le script ne gère pas les ajouts et suppressions sans remplacement, et
les caractères sensibles XML doivent être échappés
manuellement\footnote{En raison de la présence de balises XML dans les
  fichiers \texttt{.txt} avant le \emph{diff}, il n'était pas possible
  d'automatiser cette étape dans le script}. Ce script gagnerait donc à
être amélioré (notamment avec des regex plus solides), voire à être
remplacé par une approche différente et une réorganisation plus
cohérente de la chaîne de traitement. En l'état, il a surtout permis
d'accélérer le travail. Cette reprise manuelle est également l'occasion
de compléter le teiHeader en renseignant les métadonnées propres à
chaque article, et d'ajouter un attribut \texttt{@corresp} à chaque
élément \texttt{\textless{}choice\textgreater{}}.

Une transformation XSLT est ensuite appliquée pour ajouter les
identifiants à chaque élément \texttt{\textless{}choice\textgreater{}}
et ses enfants.

Grâce à ces identifiants, un prototype XForms permet la consultation et
la validation des corrections selon un parcours en plusieurs étapes. Un
premier onglet dédié à la visualisation des changements dits
``mécaniques'', sans possibilité d'action\,: il s'agit uniquement d'en
prendre connaissance. Un deuxième onglet permet de consulter les
modifications dites ``énonciatives'' pour les accepter ou les refuser.
Cette validation repose sur la bascule entre les éléments
\texttt{\textless{}sic\textgreater{}} et
\texttt{\textless{}corr\textgreater{}}\,: la version retenue est placée
dans \texttt{\textless{}sic\textgreater{}}, tandis que l'autre est
conservée dans \texttt{\textless{}corr\textgreater{}}\footnote{Ce
  stratagème constitue un abus de l'utilisation de ces balise, mais
  permet de conserver un historique des états du texte sans avoir
  recours aux éléments TEI dédiés à l'édition critique, qui relève
  plutôt d'une enquête sur la circulation des textes anciens}. Enfin, un
troisième onglet donne accès à la lecture de la version modifiée du
texte, avec une mise en exergue des changements acceptés ou refusés.

\begin{figure}
\centering
\pandocbounded{\includegraphics[keepaspectratio]{./workflow.png}}
\caption{Schéma synthétique de la pipeline}
\end{figure}

\subsection{Post-mortem}\label{post-mortem}

Ce travail, très manuel, met en évidence les difficultés liées à
l'automatisation du suivi et de la classification des révisions.

\subsubsection{Corrections non pertinentes en
masse}\label{corrections-non-pertinentes-en-masse}

Les principales corrections effectuées par Arilys et Adrien relèvent de
la correction typographique\,: remplacement d'apostrophes droites par
des apostrophes courbes, ajout d'espaces fines insécables avant
certaines ponctuations, des opérations de type
«\,chercher--remplacer\,». Ces interventions, massives et diffuses, ne
nécessitent pas d'être signalées individuellement à chaque occurrence.
Pour le présent travail, nous avons choisi de nous appuyer sur des
versions produites après ces stades de corrections typographiques
(versions identifiées ``CQ'' pour contrôle qualité), afin de réduire le
bruit.

\subsubsection{Gestion de la granularité et des changements
d'échelle}\label{gestion-de-la-granularituxe9-et-des-changements-duxe9chelle}

Pour ce présent travail, on a une granularité qui concerne le mot. Elle
est satisfaisante dans une certaine mesure, car la plupart des
corrections portent sur des mots isolés ou des groupes de mots contigus.

Cependant, des changements d'échelle apparaissent selon le type de
variation. Par exemple, lorsque la ponctuation est accolée à un mot, le
diff considère l'ensemble comme une seule unité. Dans certains cas, cela
rend compte de manière pertinente d'une correction typographique, comme
dans l'exemple suivant\,:

\begin{Shaded}
\begin{Highlighting}[]
\NormalTok{\textless{}}\KeywordTok{app}\NormalTok{\textgreater{}}
\NormalTok{    \textless{}}\KeywordTok{lem}\OtherTok{ wit=}\StringTok{"\#V1"}\NormalTok{\textgreater{}vie .\textless{}/}\KeywordTok{lem}\NormalTok{\textgreater{}}
\NormalTok{    \textless{}}\KeywordTok{rdg}\OtherTok{ wit=}\StringTok{"\#V2"}\NormalTok{\textgreater{}vie.\textless{}/}\KeywordTok{rdg}\NormalTok{\textgreater{}}
\NormalTok{\textless{}/}\KeywordTok{app}\NormalTok{\textgreater{}}
\end{Highlighting}
\end{Shaded}

Dans ce cas précis, l'encodage rend compte de manière pertinente de la
correction de la ponctuation, en fonction de ses rapports avec le texte
qui l'entoure. Le niveau de granularité adopté (le mot assorti de sa
ponctuation) semble adéquat. On pourrait cependant questionner la
pertinence de prendre en compte systématiquement la ponctuation accolée
au mot.

En revanche, dans d'autres contextes, la variation ne concerne plus un
signe ou un mot isolé, mais un ensemble textuel cohérent. C'est le cas,
par exemple, dans l'article de Snauwert, où l'ensemble des citations est
passé en italique par les éditeur·rices\,: doit-on considérer chaque
balise italique comme une micro-variation, ou traiter l'ensemble du
passage comme une seule intervention ?

Des difficultés similaires apparaissent pour certaines substitutions, où
un syntagme est remplacé par un pronom (difficile de remonter la chaîne
de référence), et pour les déplacements de segments textuels, qui ne
sont pas reconnus comme tels par l'algorithme.

Automatiser complètement la détection des interventions implique donc
des risques\,: soit une sur-fragmentation du texte, rendant le track
changes illisible, soit une agrégation trop grossière, qui efface la
nature précise des interventions. Cette tension entre précision locale
et cohérence globale constitue, selon nous, le principal défi de la
modélisation automatique des types de révision.

\subsubsection{Pertinence des
classifications}\label{pertinence-des-classifications}

Contrairement à l'édition papier, où les correcteur·rices annotaient une
épreuve destinée à l'auteur puis au métier chargé de reporter ces
changements sur la copie, ici les éditeur·rices effectuent eux-mêmes les
modifications. Le suivi strict du modèle papier est donc questionnable
dans un contexte numérique. La catégorisation qui semble la plus aidante
serait celle qui structure le dialogue avec l'auteur·ice. Or la
frontière entre corrections ``mécaniques'' et modifications
``énonciatives'' n'est pas strictement formelle. Une même opération (par
exemple une variation de ponctuation ou de casse) peut relever tantôt
d'une correction mécanique (uniformisation, conformité aux besoins de la
revue), tantôt d'un choix auctorial passant par-dessus des questions
d'uniformisation. De même, il est difficile de distinguer une simple
rectification normative d'une réécriture motivée par des considérations
stylistiques. La catégorisation dépend donc autant de l'intention de
l'intervenant·e que de la nature de l'opération.

Si certaines interventions élémentaires peuvent être détectées
automatiquement (ajouts, suppressions, substitutions au niveau d'un mot,
d'un paragraphe ou de la ponctuation), l'attribution de ces opérations à
un domaine d'expertise spécifique nécessite une interprétation humaine.
Cette classification justifie le maintien des ``petites
mains''\footnote{\citeproc{ref-melletPetitesMainsLedition2023}{\textsc{Mellet},
  Margot. {``Les petites mains de l'édition : Réflexion pour des
  environnements éditoriaux équitables, pluriels et inclusifs.''}
  {[}n.p.{]} : {[}n.p.{]}, 2023, p. 83--102. }} dans le workflow,
soulignant l'importance du rapport humain·e entre réviseur·euse et
auteur·ice.

\subsubsection{Signature ?}\label{signature}

Dans Stylo, il est difficile de déterminer précisément qui intervient à
chaque étape du processus de révision. Nous avons donc déduit la
responsabilité des changements en s'appuyant sur les titres et les
créateur·rices des versions.

Les contraintes de temps s'ajoutant à la complexité du suivi nous ont
conduite à abandonner l'idée d'introduire un attribut
\texttt{\textless{}resp\textgreater{}} à chaque modification. Ce
renoncement met toutefois en évidence une question\,: quel est l'intérêt
d'attribuer les interventions aux différents acteurs ? avec quel niveau
de granularité ? On pourrait plutôt raisonner en termes d'étapes
significatives, de rationalisation du processus de révision, plutôt que
sur chaque modification ponctuelle. Il s'agirait d'identifier des
moments pivots et de leur associer une responsabilité. Cela permettrait
de conserver un historique clair des moments clés du texte, tout en
laissant une marge de manœuvre pour le \emph{versioning}, qui ne sert
pas uniquement au suivi des révisions.

\subsection*{Biliographie}\label{biliographie}
\addcontentsline{toc}{subsection}{Biliographie}

\phantomsection\label{refs}
\begin{CSLReferences}{0}{1}
\bibitem[\citeproctext]{ref-TEIGuidelinesSimple}
{``{The TEI Guidelines - Simple Editorial Changes}.''} URL~:
\url{https://tei-c.org/release/doc/tei-p5-doc/en/html/CO.html}.

\bibitem[\citeproctext]{ref-ChicagoManualStyle}
{``The {Chicago Manual} of {Style}, 18th {Edition}.''} URL~:
\url{https://www.chicagomanualofstyle.org/book/ed18/frontmatter/toc.html}.

\bibitem[\citeproctext]{ref-beshero-bondarDocumentModelingTEI2019}
\textsc{Beshero-Bondar}, Elisa.
{``\href{https://doi.org/10.5281/ZENODO.3446154}{Document {Modeling}
with the {TEI Critical Apparatus}: {A Panel Presentation}}.''} September
2019.

\bibitem[\citeproctext]{ref-melletPetitesMainsLedition2023}
\textsc{Mellet}, Margot. {``Les petites mains de l'édition : Réflexion
pour des environnements éditoriaux équitables, pluriels et inclusifs.''}
{[}n.p.{]}. {[}n.p.{]}. 2023, p. 83--102.

\end{CSLReferences}

\end{document}
